\documentclass{article}
\usepackage{amsmath}
\usepackage[margin=1in]{geometry}
\usepackage[utf8]{inputenc}
\usepackage[T1]{fontenc}
\begin{document}

\textbf{Calculs statistiques pour le Boeuf à ragoût}

\begin{align*}
&\text{Données: } 8,44\$, 4,54\$, 2,13\$, 3,29\$ \\[1em]
&\text{1. Étendue} \\
&\text{Étendue} = \text{Max} - \text{Min} \\
&\text{Étendue} = 8,44\$ - 2,13\$ = 6,31\$ \\[1em]
&\text{2. Moyenne} \\
&\bar{x} = \frac{8,44 + 4,54 + 2,13 + 3,29}{4} = \frac{18,40}{4} = 4,60\$ \\[1em]
&\text{3. Variance} \\
&\sigma^2 = \frac{\sum(x_i - \bar{x})^2}{n} \\
&\sigma^2 = \frac{(8,44-4,60)^2 + (4,54-4,60)^2 + (2,13-4,60)^2 + (3,29-4,60)^2}{4} \\
&\sigma^2 = \frac{(3,84)^2 + (-0,06)^2 + (-2,47)^2 + (-1,31)^2}{4} \\
&\sigma^2 = \frac{14,75 + 0,0036 + 6,10 + 1,72}{4} \\
&\sigma^2 = \frac{22,57}{4} = 5,64\$^2 \\[1em]
&\text{4. Écart-Type} \\
&\sigma = \sqrt{\sigma^2} = \sqrt{5,64} = 2,37\$ \\[1em]
&\text{5. Intervalle interquartile} \\
&\text{Données ordonnées: } 2,13\$, 3,29\$, 4,54\$, 8,44\$ \\
&Q_1 = 2,71\$ \text{ (moyenne entre 2,13\$ et 3,29\$)} \\
&Q_3 = 6,49\$ \text{ (moyenne entre 4,54\$ et 8,44\$)} \\
&\text{IQR} = Q_3 - Q_1 = 6,49\$ - 2,71\$ = 3,78\$
\end{align*}

\end{document} 